\section{CAF Validation suite}
This test suite can be used to evaluate a CAF compiler
implementation to determine the extent to which it supports the
coaray-features of the Fortran 2008 standard in accordance with
ISO/IEC 1539-1:2010 (E)\cite{ftn2008}.


\subsection{Test Categories}

\subsubsection{Conformance Tests}

These tests can be found under \$(VALIDATION\_PATH)~/should\_pass.
These test the correctness of the basic constructs/semantics
of the Coarray FORTRAN. The tests mainly focus on the
verification of the correctness of contiguous and strided remote
read/write operations, the coarray syntax, the use of strided
co-subscript notation, coarrays of different data types and
image-query intrinsics.


\subsection{Confidence Tests}
These tests can be found under
\$(VALIDATION\_PATH)~/confidence\_tests. All tests in this
category verify CAF constructs whose primary purpose is to
maintain consistency among images. Such constructs aid in
avoiding race conditions. The tests include computation
intensive loops or calls to the sleep() intrinsic for
intentionally introducing delays in progress of certain images.
This enforces an increase in the likelihood of races in case of
incorrect implementations of the constructs being tested.

To determine whether, given certain values of the configuration
parameters, the tests are capable of correctly evaluating the
configuration parameters or not, every tests has a CROSS TEST
version. Such tests include the exact same code like the
original version but with certain statement deleted / replaced
with semantically incorrect sections of code. Most of the time,
the modifications to the original code is in the form of the
absence of the CAF statements which are being tested in that
program. In order to reduce redundant code, we prefer to include
the CROSS test version and the original version all in one file,
using conditional `\#ifdef' definition of the macro - CROSS\_ .

\subsection{Fault Tests}

These tests can be found under \$(VALIDATION\_PATH)/fault\_tests.
These test the correctness of the implementation of specifiers
used for handling normal and error termination. Here's an
excerpt describing these 2 types [from Page 23, section 13 of
`` ISO/IEC JTC1/SC22/WG5 N1824 ''\cite{Numrich:2005:CNF}]:

{\small  It seems natural to allow all images to continue
executing until they have all executed a stop or end program
statement, provided
  none of them encounters an error condition that may be
expected to terminate its execution. This is called normal
termination.
  On the other hand, if such an error condition occurs on one
image, the computation is flawed and it is desirable to stop
  the other images as soon as is practicable. This is called
error termination.

Normal termination occurs in three steps: initiation,
synchronization, and completion. An image
  initiates normal termination if it executes a stop or end
program statement. All images synchronize execution at the
  second step so that no image starts the completion step until
all images have finished the initiation step. The
  synchronization step allows its data to remain accessible to
the other images until they all reach the synchronization step.
  Normal termination may also be initiated during execution of a
procedure defined by a C companion processor.
}

An image initiates error termination if it executes a
statement that
  would cause the termination of a single-image program but is
not a stop or end program statement. This causes all other
  images that have not already initiated error termination to
initiate error termination. Within the performance limits of
  the processor.s ability to send signals to other images, this
propagation of error termination should be immediate. The
  exact details are intentionally left processor dependent.


\subsection{Detecting / Reporting of errors}

\begin{itemize}
\item Every test includes a module called 'crosstest'.
\item The module 'crosstest' (defined in file testmofule.f90)
includes the declaration of an integer scalar coarray called
'cross\_err' which is modified by image with rank 1 on detecting
an error. It also contains two subroutines - calc\_ori and calc.
The former returns the test result when the specific construct
is being tested. The latter is called by the cross-test version
and returns the confidence with which the original test passed.
\end{itemize}

\subsection{Specific Tests}
If the user wants to evaluate a CAF implementation in terms of
the support to specific types of tests, the names of the tests
can be specified in the file ``test\_file in the directory''
validation\textbackslash~few\_tests.

The results of the tests listed in latest\_results.log are all
subdivided into the corresponding categories.


\subsection{List of tests}

The different types of tests are listed in 
Tables~\ref{tab:conformance_files}~\ref{tab:confidence_files}~\ref{tab:fault_files}:

\begin{table}[h]
\caption{Conformance Test files in the UH - CAF Validation Tests suite}
\label{tab:conformance_files}
\begin{tabular}{|l|p{8cm}|}
\hline
File & Description \\ \hline
character\_test.f90 &   CHARACTER coarrays\\ \hline
coarray\_2.4.7.6.f90 &   similar translation of co-subscripts and subscripts\\ \hline
coarray\_4.8.R468.f90 &  reference of coarray without [] implies local object\\ \hline
coarray\_5.3.6.1.f90 &   attribute CODIMENSION + remote accesses at single integer/real boundary\\ \hline
dummyargs\_12.3.2.2c.f90 &   explicit shape, assumed size, assumed shape, allocatable dummy args\\ \hline
intrin\_13.7.126.f90 &   NUM\_IMAGES() returns the number of images launched\\ \hline
intrin\_13.7.165.f90 &   THIS\_IMAGE(), THIS\_IMAGE(coarray), THIS\_IMAGE(coarray, dim)\\ \hline
intrin\_13.7.172.f90 &   LCOBOUND(coarray) and LCOBOUND(coarray,dim)\\ \hline
intrin\_13.7.79.f90 &   IMAGE\_INDEX(coarray, subs)\\ \hline
intrin\_13.7.91.f90 &   UCOBOUND(COARRAY[, DIM, KIND])\\ \hline
intrin\_6.7.3.2.11.f90 &   ALLOCATE and DEALLOCATE act as barriers\\ \hline
item\_4.8.a.f90 &   Subobjects of a coarray is also a coarray\\ \hline
pointer\_4.5.4.6b.f90 &   association of pointer components of coarrays with local objects\\ \hline
intrin\_8.5.7d.f90 &  STOP and LOCK construct with STAT=STAT\_LOCKED specifier\\ \hline
intrin\_8.5.7e.f90 &  STOP and LOCK construct with STAT=STAT\_LOCKED\_OTHER\_IMAGE specifier\\ \hline
intrin\_8.5.7f.f90 &  STOP and LOCK construct with STAT=STAT\_UNLOCKED specifier\\ \hline
derived\_4.5.4.f90 &  (non-)coarray COMPONENTS of (non-)coarray derived types\\ \hline
\end{tabular}
\end{table}


\begin{table}[h]
\caption{Confidence Test files in the UH - CAF Validation Tests suite}
\label{tab:confidence_files}
\begin{tabular}{|l|p{8cm}|}
\hline
File & Description \\ \hline
atomic\_8.5.2.f90 & Atomic subroutines\\ \hline
critical\_8.1.5.f90 & CRITICAL - END CRITICAL sections\\ \hline
intrin\_8.5.6.f90  & LOCK \& UNLOCK without STAT specifier\\ \hline
sync\_8.5.3.f90  & SYNC ALL without STAT specifier\\ \hline
sync\_8.5.4a.f90 & SYNC IMAGES(arr) paired with SYNC IMAGES(*)\\ \hline
sync\_8.5.4b.f90 & call to SYNC IMAGES(arr), should not behave like SYNC ALL\\ \hline
\hline
\end{tabular}
\end{table}

\begin{table}[h]
\caption{Fault (tolerance) Test files in the UH - CAF Validation Tests suite}
\label{tab:fault_files}
\begin{tabular}{|l|p{8cm}|}
\hline
File & Description \\ \hline
sync\_8.5.7a.f90 & STOP and SYNC ALL with STAT=STAT\_STOPPED\_IMAGE specifier\\ \hline
sync\_8.5.7b.f90 & STOP and SYNC IMAGES(arr) with STAT=STAT\_STOPPED\_IMAGE specifier\\ \hline
sync\_8.5.7c.f90 &  STOP and SYNC IMAGES(*) with STAT=STAT\_STOPPED\_IMAGE specifier\\ \hline
\end{tabular}
\end{table}

\subsection{Configuration Parameters}
Table~\ref{tab:params} lists the different options that need to be
set for the test-suite. These options can be initialized in the
CONFIG file in the ``config'' directory. The make.def version of
this file is auto-generated by running the script
config2makedef.sh in the same directory.

\begin{table}[tbh!]
\caption{List of configuration parameters. Here \emph{test\_type}= CONF or FEATURE or FEW or FAULT }
\label{tab:params}
\begin{tabular}{|c|p{6cm}|p{2cm}|}
\hline
Parameter & Description &   Compiler specific(y/n)  \\ \hline
BIN\_PATH   & The path to dump all the executables.  &  NO \\\hline
NPROCS &  Number of images to launch  & NO \\\hline
NITER &  Number of times the test is repeated & NO \\\hline
SLEEP &  This is used to intentionally slow down certain images to cause race conditions while testing certain constructs & NO \\\hline
TIMEOUT & This parameter is passed to the perl script call timedexec.pl which ends processes which exceed the given execution time. This is helpful for backing out while executing tests which deadlock due to incorrect implementation &  NO \\\hline
COMPILER &  compiler name  &  YES \\\hline
FC &  command to invoke the compiler &  YES \\\hline
FFLAGS &  Flags passed to the compiler. The necessary flags include the options to enable the macro preprocessor and define the macros - NPROCS, NITER and SLEEP. &  YES \\\hline
LAUNCHER &  command to launch multiple images  & YES \\\hline
EXEC\_OPTIONS &   Flags passed to the launcher after the executable name. Not so common.  & YES \\\hline
FFLAGS\_CROSS &  Flags passed to the compiler while executing the cross tests. Generally the value include all the options listed for FFLAGS, plus the flag to define the CROSS\_ macro in the tests.  & YES \\\hline
\emph{test\_type}\_COMPILE\_PATH &  Path to dump the messages generated by the compiler.&  NO \\\hline
\emph{test\_type}\_EXEC\_PATH &  Path to dump the output of the executables / the messages generated by the compiler or runtime.&  NO \\\hline
\emph{test\_type}\_LOG\_PATH &  Path to dump the output of the test results.& NO \\\hline
\end{tabular}
\end{table}

\subsection{APPENDIX}

Tables~\ref{tab:conformance}~\ref{tab:confidence}~\ref{tab:fault} list the extent of support of coarrays in different CAF compiler implementations as detected by the UH CAF validation test suite:

\begin{table}[tbh!]
\caption{Results of Conformance tests}
\label{tab:conformance}
\begin{tabular}{|l|p{8cm}|p{1.5cm}|p{1.5cm}|}
\hline
SPEC\_IDX              &  DESCRIPTION  & OpenUH & Intel \\\hline
character\_test        &  CHARACTER coarrays & YES & YES \\\hline
coarray\_2.4.7.6       &  similar translation of co-subscripts and subscripts & YES & YES \\\hline
coarray\_4.8.R468      &  reference of coarray without [] implies local object & YES & YES \\\hline
coarray\_5.3.6.1       &  attribute CODIMENSION + remote accesses at single integer/real boundary & YES & Execution times out \\\hline
dummyargs\_12.3.2.2c   &  explicit shape, assumed size, assumed shape, allocatable dummy arguments & YES & YES \\\hline
intrin\_13.7.126       &  NUM\_IMAGES() returns the number of images launched & YES & YES \\\hline
intrin\_13.7.165       &  THIS\_IMAGE(), THIS\_IMAGE(coarray), THIS\_IMAGE(coarray, dim) & YES & YES \\\hline
intrin\_13.7.172       &  LCOBOUND(coarray) and LCOBOUND(coarray,dim) & YES & YES \\\hline
intrin\_13.7.79        &  IMAGE\_INDEX(coarray, subs) & YES & YES \\\hline
intrin\_13.7.91        &  UCOBOUND(COARRAY[, DIM, KIND]) & YES & YES \\\hline
intrin\_6.7.3.2.11     &  ALLOCATE and DEALLOCATE act as barriers & YES & YES \\\hline
intrin\_8.5.7d         &  STOP and LOCK construct with STAT=STAT\_LOCKED specifier & YES & YES \\\hline
intrin\_8.5.7e         &  STOP and LOCK construct with STAT=STAT\_LOCKED\_OTHER\_IMAGE specifier & Execution times out & Execution times out \\\hline
intrin\_8.5.7f         &  STOP and LOCK construct with STAT=STAT\_UNLOCKED specifier & YES & Execution times out \\\hline
item\_4.8.a            &  subobjects if a coarray is also a coarray & YES & YES \\\hline
pointer\_4.5.4.6b      &  association of pointer components of coarrays with local objects & YES & Execution times out \\\hline
\end{tabular}
\end{table}

\begin{table}[tbh!]
\caption{Results of Confidence tests}
\label{tab:confidence}
\begin{tabular}{|l|p{6cm}|p{2cm}|p{2cm}|}
\hline
SPEC\_IDX          &  DESCRIPTION  & OpenUH & Intel \\\hline
atomic\_8.5.2      &  Atomic subroutines & Passes with 0\% confidence & Fails compilation \\\hline
critical\_8.1.5    &  CRITICAL - END CRITICAL sections & YES & Passes with 0\% confidence \\\hline
intrin\_8.5.6      &  LOCK \& UNLOCK without STAT specifier & YES & Fails compilation \\\hline
sync\_8.5.3        &  SYNC ALL without STAT specifier & YES & Passes with 0\% confidence \\\hline
sync\_8.5.4a       &  SYNC IMAGES(arr) paired with SYNC IMAGES(*) & YES & Fails execution \\\hline
sync\_8.5.4b       &  call to SYNC IMAGES(arr), should not behave like SYNC ALL & YES & Fails execution \\\hline
\end{tabular}
\end{table}


\begin{table}[tbh!]
\caption{Results of Fault tests}
\label{tab:fault}
\begin{tabular}{|l|p{6cm}|p{2cm}|p{2cm}|}
\hline
SPEC\_IDX          &      DESCRIPTION  & OpenUH & Intel \\\hline
sync\_8.5.7a       &      STOP and SYNC ALL with STAT=STAT\_STOPPED\_IMAGE specifier & YES & Fails execution \\\hline
sync\_8.5.7b       &      STOP and SYNC IMAGES(arr) with STAT=STAT\_STOPPED\_IMAGE specifier & YES & YES \\\hline
sync\_8.5.7c       &      STOP and SYNC IMAGES(*) with STAT=STAT\_STOPPED\_IMAGE specifier & YES & Fails execution \\\hline
\end{tabular}
\end{table}

